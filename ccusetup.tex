% !TeX root = ./main.tex

% --------------------------------------------------
% 資訊設定(Information Configs)
% --------------------------------------------------

\ccusetup{
  university*   = {National Chung Cheng University},
  university    = {國~~~~立~~~~中~~~~正~~~~大~~~~學},
  college       = {工學院},
  college*      = {College of Engineering},
  institute     = {資訊工程研究所},
  institute*    = {Computer Science and Information Engineering}, % 請確認您的科系為Department of xxx或Institute of xxx
  title         = {中文題目},
  title*        = {English title},
  author        = {呂偉嘉},
  author*       = {Wei-Jia Lyu},
  advisor       = {張榮貴},
  advisor*      = {Rong-Guey Chang},
  date          = {一一三~年~六~月},                      % ~ 表示空格
  keywords      = {LaTeX, 中文, 論文, 模板},
  keywords*     = {LaTeX, CJK, Thesis, Template},
}

% --------------------------------------------------
% 加載套件(Include Packages)
% --------------------------------------------------

\usepackage{amsmath, amsthm, amssymb}   % 數學環境
\usepackage{bm}                         % 數學符號加粗
\usepackage{ulem}                       % 下劃線、雙下劃線與波浪紋效果
\usepackage{booktabs}                   % 改善表格設置
\usepackage{multirow}                   % 合併儲存格
\usepackage{diagbox}                    % 插入表格反斜線
\usepackage{array}                      % 調整表格高度
\usepackage{longtable}                  % 支援跨頁長表格
\usepackage{paralist}                   % 列表環境
\usepackage{zhnumber}                   % 中文數字
\usepackage{algorithm, algpseudocode}   % 演算法
\usepackage{graphics, graphicx}         % 圖片
\usepackage{rotating}                   % 旋轉圖片
\usepackage{notoccite}                  % 避免文獻引用標號順序錯亂

% 下列產生亂字的pkg可刪除
\usepackage{lipsum}                     % 英文亂字
\usepackage{zhlipsum}                   % 中文亂字
