\documentclass[
  twoside,
  openright,
  degree    = master,                       % degree   = master  | doctor
  language  = chinese,                      % language = chinese | english
  fontset   = template,                     % fontset  = default | template
%   draft                                   % LaTex draft mode (images will change to box)
]{ccuthesis}

% !TeX root = ./main.tex

% --------------------------------------------------
% 資訊設定(Information Configs)
% --------------------------------------------------

\ccusetup{
  university*   = {National Chung Cheng University},
  university    = {國 \quad 立 \quad 中 \quad 正 \quad 大 \quad 學},
  college       = {工學院},
  college*      = {College of Engineering},
  institute     = {資訊工程研究所},
  institute*    = {Computer Science and Information Engineering}, % 請確認您的科系為Department of xxx或Institute of xxx
  title         = {中文題目},
  title*        = {English title},
  author        = {呂偉嘉},
  author*       = {Wei-Jia Lyu},
  advisor       = {張榮貴},
  advisor*      = {Rong-Guey Chang},
  date          = {一一三年六月},                      % ~ 表示空格
  keywords      = {LaTeX, 中文, 論文, 模板},
  keywords*     = {LaTeX, CJK, Thesis, Template},
}

% --------------------------------------------------
% 加載套件(Include Packages)
% --------------------------------------------------

\usepackage{amsmath, amsthm, amssymb}   % 數學環境
\usepackage{bm}                         % 數學符號加粗
\usepackage{ulem}                       % 下劃線、雙下劃線與波浪紋效果
\usepackage{booktabs}                   % 改善表格設置
\usepackage{multirow}                   % 合併儲存格
\usepackage{diagbox}                    % 插入表格反斜線
\usepackage{array}                      % 調整表格高度
\usepackage{longtable}                  % 支援跨頁長表格
\usepackage{paralist}                   % 列表環境
\usepackage{zhnumber}                   % 中文數字
\usepackage{algorithm, algpseudocode}   % 演算法
\usepackage{graphics, graphicx}         % 圖片
\usepackage{rotating}                   % 旋轉圖片
\usepackage{notoccite}                  % 避免文獻引用標號順序錯亂

% 下列產生亂字的pkg可刪除
\usepackage{lipsum}                     % 英文亂字
\usepackage{zhlipsum}                   % 中文亂字


\begin{document}
	% 封面 Cover
	% 若需要在封面加上(初稿)字樣,請在\makecover後加上{draft}
	% 
	% \makecover{draft}                               % 有(初稿)字樣的論文封面(Cover with Draft)
	\makecover                                        % 論文封面(Cover)

	% 口試審定 Verification Letter
	% \makeverification會生成空白的審定書
	% 若有審定書pdf檔案,可用\renderverification{檔案路徑}渲染審定書
	% 
	% \makeverification                               % 口試委員審定書(Verification Letter)
	%\renderverification{frontpages/verification}      % 渲染口試委員審定書(Render Verification Letter)

	% 加入浮水印 Watermark
	% 若需要再封面加上浮水印,請將\makewatermark{watermark}放在\makecover之前
	% 
	\makewatermark{watermark}                         % 加入浮水印(Watermark)

	% 致謝與論文摘要 Acknowledgement and Abstract
	\frontmatter
	% !TeX root = ../main.tex

\begin{acknowledgement}

致謝致謝致謝致謝致謝致謝致謝致謝致謝致謝致謝致謝致謝致謝致謝

\end{acknowledgement}                % 致謝(Acknowledgement)
	\input{frontpages/abstract}                       % 摘要(Abstract)
	
	% 生成目錄與符號列表 Tables of Contents and Denotation
	\maketableofcontents                              % 目錄(Table of Contents)
	\makelistoffigures                                % 圖目錄(List of Figures)
	\makelistoftables                                 % 表目錄(List of Tables)
	\input{frontpages/denotation}                     % 符號列表(Denotation)

	% 論文內容 Contents of Thesis
	\mainmatter
	% !TeX root = ../main.tex

% ===================== 緒論 ===================== %
\section{簡介}

% ------------------- 1.1------------------- %
\subsection{研究背景}


% ------------------- 1.2------------------- %
\subsection{研究動機}

% ------------------- 1.3------------------- %
\subsection{研究目的}
                   % 緒論(Introduction)
	\input{sections/02related_work}                   % 文獻探討(Related Work)
	\input{sections/03method}                         % 研究方法(Method)
	\input{sections/04experiments}                    % 研究結果(Experiments)
	\input{sections/05conclusion}                     % 結論(Conclusion)

	% 參考文獻 References
	\refmatter
	\nocite{*}
	\bibliographystyle{IEEEtran}                      % 參考文獻格式(References Style)
	\bibliography{backpages/references}               % 參考文獻資料庫(References Database)

	% 附錄 Appendix
	\input{backpages/appendix}                        % 附錄(Appendix)
\end{document}
